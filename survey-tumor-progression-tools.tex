\documentclass{article}
\usepackage[utf8]{inputenc}

\title{Comparison of Tools for Tumor Progression Inference}
\author{}
\date{}

\begin{document}

\maketitle

\section{Introduction}

\section{Tools}
Io pensavo a questa selezione di tools (se ne conoscete altri aggiungete pure):

Bulk Sequencing:
- CITUP https://academic.oup.com/bioinformatics/article/31/9/1349/200674
- Lichee https://genomebiology.biomedcentral.com/articles/10.1186/s13059-015-0647-8
- SPRUCE (?) https://www.cell.com/cell-systems/abstract/S2405-4712(16)30221-6

Single Cell:
- SCITE https://genomebiology.biomedcentral.com/articles/10.1186/s13059-016-0936-x
- SiFit https://genomebiology.biomedcentral.com/articles/10.1186/s13059-017-1311-2

Bulk + Single Cell (?):
- B-SCITE https://www.biorxiv.org/content/early/2017/12/15/234914
- ddclone https://genomebiology.biomedcentral.com/articles/10.1186/s13059-017-1169-3

Più i nostri tools

\section{Datasets}

Per rispondere a Simone, noi abbiamo buoni algoritmi per la generazione di dataset binari sia single-cell che multi-region, con vari modelli di rumore/missing data, ecc. Per le altre tipologie di dato/sampling bisogna capire che fare, sicuramente non reinventare la ruota, ma magari affidarsi a tecniche già esistenti.
\textbf{decidere per la parte sopra}

Infine, per rispondere a Gianluca, il dato di TRACERx, che abbiamo già analizzato con i nostri tool in via preliminare, è un ottimo punto di partenza per testare le varie tecniche dal punto di vista applicativo, ma è un dato con peculiarità ben precise: dato bulk multi region di pazienti multipli di lung cancer. Non tutti gli algoritmi possono essere utilizzati con quel tipo di dato e anche l’output stesso può essere consistentemente diverso a seconda del goal (e.g., un modello per paziente? Un modello unico per la coorte in cui evidenziare regolarità? Filogenia/mutational tree/clonal tree? ecc.).

\section{Experiment sketch}

A mio parere è opportuno partizionare il tutto in casi relativamente
omogenei ed evitare confronti fra casi. In altre parole, confronterei
separatamente bulk da single cell.

Per semplicità, almeno in una prima fase, mi concentrerei solo su
singolo paziente: credo sia già abbastanza. Inoltre su più pazienti
entrano in gioco una serie di fattori confondenti che sono delicati da
trattare.

Personalmente mi concentrerei su avere nodi che sono celle/cloni (dal
mio punto di vista la distinzione fra questi è artificiale), anche per
avere maggiore supporto, dal punto di vista della valutazione, dalla
letteratura in filogenetica.



\end{document}
